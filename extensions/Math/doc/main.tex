% !TeX spellcheck = en_US
\documentclass[a4paper,12pt]{report}
\usepackage{amsmath}
\usepackage{amsfonts}
\usepackage{amssymb}
\usepackage{bbold}
\usepackage{cancel}
\usepackage{texvc}
\usepackage{mhchem}
\usepackage{stix} %intbar
%\usepackage{MnSymbol} %strokeint
\usepackage{arcs}
\usepackage{url}
\usepackage{parskip}
\author{Moritz Schubotz}
\title{Technical details on texvcjs}
\begin{document}
\maketitle
\chapter{Technical details on texvc identifier extraction}
\section{Introduction}
This chapter describes which mathematical symbols are identified as identifiers.
In general every single Latin letter [a-zA-Z] is regarded as identifier.
In addition, we accept multi-letter-subscripts that match [0-9a-zA-Z]+, such as $a_0$ but also $\varepsilon_{ijk}$.
Moreover, the Literals described in section~\ref{sc.lit}, and the Identifier variants (section~\ref{sc.var}) are supported.
\section{Literals}\label{sc.lit}
The following literals are supported:

\texttt{\textbackslash Bbbk} is rendered as $\Bbbk$


\texttt{\textbackslash Delta} is rendered as $\Delta$


\texttt{\textbackslash Finv} is rendered as $\Finv$


\texttt{\textbackslash Game} is rendered as $\Game$


\texttt{\textbackslash Gamma} is rendered as $\Gamma$


\texttt{\textbackslash Lambda} is rendered as $\Lambda$


\texttt{\textbackslash Omega} is rendered as $\Omega$


\texttt{\textbackslash P} is rendered as $\P$


\texttt{\textbackslash Phi} is rendered as $\Phi$


\texttt{\textbackslash Pi} is rendered as $\Pi$


\texttt{\textbackslash Psi} is rendered as $\Psi$


\texttt{\textbackslash S} is rendered as $\S$


\texttt{\textbackslash Sigma} is rendered as $\Sigma$


\texttt{\textbackslash Theta} is rendered as $\Theta$


\texttt{\textbackslash Xi} is rendered as $\Xi$


\texttt{\textbackslash aleph} is rendered as $\aleph$


\texttt{\textbackslash alpha} is rendered as $\alpha$


\texttt{\textbackslash amalg} is rendered as $\amalg$


\texttt{\textbackslash backepsilon} is rendered as $\backepsilon$


\texttt{\textbackslash beta} is rendered as $\beta$


\texttt{\textbackslash beth} is rendered as $\beth$


\texttt{\textbackslash chi} is rendered as $\chi$


\texttt{\textbackslash complement} is rendered as $\complement$


\texttt{\textbackslash daleth} is rendered as $\daleth$


\texttt{\textbackslash delta} is rendered as $\delta$


\texttt{\textbackslash digamma} is rendered as $\digamma$


\texttt{\textbackslash ell} is rendered as $\ell$


\texttt{\textbackslash epsilon} is rendered as $\epsilon$


\texttt{\textbackslash eta} is rendered as $\eta$


\texttt{\textbackslash eth} is rendered as $\eth$


\texttt{\textbackslash flat} is rendered as $\flat$


\texttt{\textbackslash gamma} is rendered as $\gamma$


\texttt{\textbackslash gimel} is rendered as $\gimel$


\texttt{\textbackslash hslash} is rendered as $\hslash$


\texttt{\textbackslash imath} is rendered as $\imath$


\texttt{\textbackslash intercal} is rendered as $\intercal$


\texttt{\textbackslash iota} is rendered as $\iota$


\texttt{\textbackslash jmath} is rendered as $\jmath$


\texttt{\textbackslash kappa} is rendered as $\kappa$


\texttt{\textbackslash lambda} is rendered as $\lambda$


\texttt{\textbackslash mho} is rendered as $\mho$


\texttt{\textbackslash mu} is rendered as $\mu$


\texttt{\textbackslash natural} is rendered as $\natural$


\texttt{\textbackslash nu} is rendered as $\nu$


\texttt{\textbackslash omega} is rendered as $\omega$


\texttt{\textbackslash phi} is rendered as $\phi$


\texttt{\textbackslash pi} is rendered as $\pi$


\texttt{\textbackslash pitchfork} is rendered as $\pitchfork$


\texttt{\textbackslash psi} is rendered as $\psi$


\texttt{\textbackslash rho} is rendered as $\rho$


\texttt{\textbackslash sigma} is rendered as $\sigma$


\texttt{\textbackslash tau} is rendered as $\tau$


\texttt{\textbackslash theta} is rendered as $\theta$


\texttt{\textbackslash top} is rendered as $\top$


\texttt{\textbackslash varepsilon} is rendered as $\varepsilon$


\texttt{\textbackslash varkappa} is rendered as $\varkappa$


\texttt{\textbackslash varnothing} is rendered as $\varnothing$


\texttt{\textbackslash varphi} is rendered as $\varphi$


\texttt{\textbackslash varpi} is rendered as $\varpi$


\texttt{\textbackslash varrho} is rendered as $\varrho$


\texttt{\textbackslash varsigma} is rendered as $\varsigma$


\texttt{\textbackslash vartheta} is rendered as $\vartheta$


\texttt{\textbackslash wp} is rendered as $\wp$


\texttt{\textbackslash xi} is rendered as $\xi$


\texttt{\textbackslash zeta} is rendered as $\zeta$



\section{Identifier variants}\label{sc.var}
The following variants are supported\footnote{Note that \texttt{\textbackslash mathcal} is not available for lowercase Latin letters.}:

\texttt{\textbackslash Bbb} applied on $x,X$ is rendered as $\Bbb{x},\Bbb{X}$


\texttt{\textbackslash acute} applied on $x,X$ is rendered as $\acute{x},\acute{X}$


\texttt{\textbackslash bar} applied on $x,X$ is rendered as $\bar{x},\bar{X}$


\texttt{\textbackslash bcancel} applied on $x,X$ is rendered as $\bcancel{x},\bcancel{X}$


\texttt{\textbackslash bmod} applied on $x,X$ is rendered as $\bmod{x},\bmod{X}$


\texttt{\textbackslash bold} applied on $x,X$ is rendered as $\bold{x},\bold{X}$


\texttt{\textbackslash boldsymbol} applied on $x,X$ is rendered as $\boldsymbol{x},\boldsymbol{X}$


\texttt{\textbackslash breve} applied on $x,X$ is rendered as $\breve{x},\breve{X}$


\texttt{\textbackslash cancel} applied on $x,X$ is rendered as $\cancel{x},\cancel{X}$


\texttt{\textbackslash check} applied on $x,X$ is rendered as $\check{x},\check{X}$


\texttt{\textbackslash ddot} applied on $x,X$ is rendered as $\ddot{x},\ddot{X}$


\texttt{\textbackslash dot} applied on $x,X$ is rendered as $\dot{x},\dot{X}$


\texttt{\textbackslash emph} applied on $x,X$ is rendered as $\emph{x},\emph{X}$


\texttt{\textbackslash grave} applied on $x,X$ is rendered as $\grave{x},\grave{X}$


\texttt{\textbackslash hat} applied on $x,X$ is rendered as $\hat{x},\hat{X}$


\texttt{\textbackslash mathbb} applied on $x,X$ is rendered as $\mathbb{x},\mathbb{X}$


\texttt{\textbackslash mathbf} applied on $x,X$ is rendered as $\mathbf{x},\mathbf{X}$


\texttt{\textbackslash mathbin} applied on $x,X$ is rendered as $\mathbin{x},\mathbin{X}$


\texttt{\textbackslash mathcal} applied on $x,X$ is rendered as $\mathcal{x},\mathcal{X}$


\texttt{\textbackslash mathclose} applied on $x,X$ is rendered as $\mathclose{x},\mathclose{X}$


\texttt{\textbackslash mathfrak} applied on $x,X$ is rendered as $\mathfrak{x},\mathfrak{X}$


\texttt{\textbackslash mathit} applied on $x,X$ is rendered as $\mathit{x},\mathit{X}$


\texttt{\textbackslash mathop} applied on $x,X$ is rendered as $\mathop{x},\mathop{X}$


\texttt{\textbackslash mathopen} applied on $x,X$ is rendered as $\mathopen{x},\mathopen{X}$


\texttt{\textbackslash mathord} applied on $x,X$ is rendered as $\mathord{x},\mathord{X}$


\texttt{\textbackslash mathpunct} applied on $x,X$ is rendered as $\mathpunct{x},\mathpunct{X}$


\texttt{\textbackslash mathrel} applied on $x,X$ is rendered as $\mathrel{x},\mathrel{X}$


\texttt{\textbackslash mathrm} applied on $x,X$ is rendered as $\mathrm{x},\mathrm{X}$


\texttt{\textbackslash mathsf} applied on $x,X$ is rendered as $\mathsf{x},\mathsf{X}$


\texttt{\textbackslash mathtt} applied on $x,X$ is rendered as $\mathtt{x},\mathtt{X}$


\texttt{\textbackslash overleftarrow} applied on $x,X$ is rendered as $\overleftarrow{x},\overleftarrow{X}$


\texttt{\textbackslash overleftrightarrow} applied on $x,X$ is rendered as $\overleftrightarrow{x},\overleftrightarrow{X}$


\texttt{\textbackslash overline} applied on $x,X$ is rendered as $\overline{x},\overline{X}$


\texttt{\textbackslash overrightarrow} applied on $x,X$ is rendered as $\overrightarrow{x},\overrightarrow{X}$


\texttt{\textbackslash textbf} applied on $x,X$ is rendered as $\textbf{x},\textbf{X}$


\texttt{\textbackslash textit} applied on $x,X$ is rendered as $\textit{x},\textit{X}$


\texttt{\textbackslash textrm} applied on $x,X$ is rendered as $\textrm{x},\textrm{X}$


\texttt{\textbackslash textsf} applied on $x,X$ is rendered as $\textsf{x},\textsf{X}$


\texttt{\textbackslash texttt} applied on $x,X$ is rendered as $\texttt{x},\texttt{X}$


\texttt{\textbackslash tilde} applied on $x,X$ is rendered as $\tilde{x},\tilde{X}$


\texttt{\textbackslash underline} applied on $x,X$ is rendered as $\underline{x},\underline{X}$


\texttt{\textbackslash vec} applied on $x,X$ is rendered as $\vec{x},\vec{X}$


\texttt{\textbackslash widehat} applied on $x,X$ is rendered as $\widehat{x},\widehat{X}$


\texttt{\textbackslash widetilde} applied on $x,X$ is rendered as $\widetilde{x},\widetilde{X}$


\texttt{\textbackslash xcancel} applied on $x,X$ is rendered as $\xcancel{x},\xcancel{X}$


\texttt{\textbackslash xleftarrow} applied on $x,X$ is rendered as $\xleftarrow{x},\xleftarrow{X}$


\texttt{\textbackslash xrightarrow} applied on $x,X$ is rendered as $\xrightarrow{x},\xrightarrow{X}$



\chapter{List of all commands supported}\label{ch:all}
Chapter~\ref{ch:all} lists all commands allowed by texvcjs.

\section{ Group \texttt{big\textunderscore literals}}

\texttt{\textbackslash Big} is rendered as $\Big($

\texttt{\textbackslash Bigg} is rendered as $\Bigg($

\texttt{\textbackslash Biggl} is rendered as $\Biggl($

\texttt{\textbackslash Biggr} is rendered as $\Biggr($

\texttt{\textbackslash Bigl} is rendered as $\Bigl($

\texttt{\textbackslash Bigr} is rendered as $\Bigr($

\texttt{\textbackslash big} is rendered as $\big($

\texttt{\textbackslash bigg} is rendered as $\bigg($

\texttt{\textbackslash biggl} is rendered as $\biggl($

\texttt{\textbackslash biggr} is rendered as $\biggr($

\texttt{\textbackslash bigl} is rendered as $\bigl($

\texttt{\textbackslash bigr} is rendered as $\bigr($

\section{ Group \texttt{box\textunderscore functions}}

\texttt{\textbackslash hbox} is rendered as $\hbox{x}$

\texttt{\textbackslash mbox} is rendered as $\mbox{x}$

\texttt{\textbackslash text} is rendered as $\text{x}$

\texttt{\textbackslash vbox} is rendered as $\vbox{x}$

\section{ Group \texttt{color\textunderscore function}}

\texttt{\textbackslash color} is rendered as $\color{red}{red}$

\texttt{\textbackslash pagecolor} is not rendered.

\section{ Group \texttt{declh\textunderscore function}}

\texttt{\textbackslash bf} is rendered as $\bf$

\texttt{\textbackslash cal} is rendered as $\cal$

\texttt{\textbackslash it} is rendered as $\it$

\texttt{\textbackslash rm} is rendered as $\rm$

\section{ Group \texttt{definecolor\textunderscore function}}

\texttt{\textbackslash definecolor} is rendered as $\definecolor{mycolor}{cmyk}{.4,1,1,0}$

\section{ Group \texttt{fun\textunderscore ar1}}

\texttt{\textbackslash acute} is rendered as $\acute{x}$

\texttt{\textbackslash bar} is rendered as $\bar{x}$

\texttt{\textbackslash bcancel} is rendered as $\bcancel{x}$

\texttt{\textbackslash bmod} is rendered as $\bmod{x}$

\texttt{\textbackslash boldsymbol} is rendered as $\boldsymbol{x}$

\texttt{\textbackslash breve} is rendered as $\breve{x}$

\texttt{\textbackslash cancel} is rendered as $\cancel{x}$

\texttt{\textbackslash check} is rendered as $\check{x}$

\texttt{\textbackslash ddot} is rendered as $\ddot{x}$

\texttt{\textbackslash dot} is rendered as $\dot{x}$

\texttt{\textbackslash emph} is rendered as $\emph{x}$

\texttt{\textbackslash grave} is rendered as $\grave{x}$

\texttt{\textbackslash hat} is rendered as $\hat{x}$

\texttt{\textbackslash hphantom} is rendered as $\hphantom{x}$





\texttt{\textbackslash mathcal} is rendered as $\mathcal{x}$

\texttt{\textbackslash mathclose} is rendered as $\mathclose{x}$

\texttt{\textbackslash mathfrak} is rendered as $\mathfrak{x}$

\texttt{\textbackslash mathit} is rendered as $\mathit{x}$

\texttt{\textbackslash mathopen} is rendered as $\mathopen{x}$

\texttt{\textbackslash mathord} is rendered as $\mathord{x}$

\texttt{\textbackslash mathpunct} is rendered as $\mathpunct{x}$

\texttt{\textbackslash mathsf} is rendered as $\mathsf{x}$

\texttt{\textbackslash mathtt} is rendered as $\mathtt{x}$





\texttt{\textbackslash overleftarrow} is rendered as $\overleftarrow{x}$

\texttt{\textbackslash overleftrightarrow} is rendered as $\overleftrightarrow{x}$

\texttt{\textbackslash overline} is rendered as $\overline{x}$

\texttt{\textbackslash overrightarrow} is rendered as $\overrightarrow{x}$

\texttt{\textbackslash phantom} is rendered as $\phantom{x}$

\texttt{\textbackslash pmod} is rendered as $\pmod{x}$





\texttt{\textbackslash sqrt} is rendered as $\sqrt{x}$

\texttt{\textbackslash textbf} is rendered as $\textbf{x}$

\texttt{\textbackslash textit} is rendered as $\textit{x}$

\texttt{\textbackslash textrm} is rendered as $\textrm{x}$

\texttt{\textbackslash textsf} is rendered as $\textsf{x}$

\texttt{\textbackslash texttt} is rendered as $\texttt{x}$

\texttt{\textbackslash tilde} is rendered as $\tilde{x}$

\texttt{\textbackslash underline} is rendered as $\underline{x}$

\texttt{\textbackslash vec} is rendered as $\vec{x}$

\texttt{\textbackslash vphantom} is rendered as $\vphantom{x}$

\texttt{\textbackslash widehat} is rendered as $\widehat{x}$

\texttt{\textbackslash widetilde} is rendered as $\widetilde{x}$

\texttt{\textbackslash xcancel} is rendered as $\xcancel{x}$

\section{ Group \texttt{fun\textunderscore ar1nb}}

\texttt{\textbackslash mathbb} is rendered as $\mathbb{x}$

\texttt{\textbackslash mathbf} is rendered as $\mathbf{x}$

\texttt{\textbackslash mathbin} is rendered as $\mathbin{x}$

\texttt{\textbackslash mathop} is rendered as $\mathop{x}$

\texttt{\textbackslash mathrel} is rendered as $\mathrel{x}$

\texttt{\textbackslash mathrm} is rendered as $\mathrm{x}$

\texttt{\textbackslash operatorname} is rendered as $\operatorname{x}$

\texttt{\textbackslash overarc} is rendered as $\overarc{x}$

\texttt{\textbackslash overbrace} is rendered as $\overbrace{x}$

\texttt{\textbackslash underbrace} is rendered as $\underbrace{x}$

\texttt{\textbackslash xleftarrow} is rendered as $\xleftarrow{x}$

\texttt{\textbackslash xrightarrow} is rendered as $\xrightarrow{x}$

\section{ Group \texttt{fun\textunderscore ar1opt}}



\texttt{\textbackslash sqrt} is rendered as $\sqrt{x}$

\texttt{\textbackslash xleftarrow} is rendered as $\xleftarrow{x}$

\texttt{\textbackslash xrightarrow} is rendered as $\xrightarrow{x}$

\section{ Group \texttt{fun\textunderscore ar2}}

\texttt{\textbackslash binom} applied on ${x}{x}$ is rendered as $\binom{x}{x}$

\texttt{\textbackslash cancelto} applied on ${x}{x}$ is rendered as $\cancelto{x}{x}$

\texttt{\textbackslash cfrac} applied on ${x}{x}$ is rendered as $\cfrac{x}{x}$

\texttt{\textbackslash dbinom} applied on ${x}{x}$ is rendered as $\dbinom{x}{x}$

\texttt{\textbackslash dfrac} applied on ${x}{x}$ is rendered as $\dfrac{x}{x}$

\texttt{\textbackslash frac} applied on ${x}{x}$ is rendered as $\frac{x}{x}$





\texttt{\textbackslash overset} applied on ${x}{x}$ is rendered as $\overset{x}{x}$



\texttt{\textbackslash stackrel} applied on ${x}{x}$ is rendered as $\stackrel{x}{x}$

\texttt{\textbackslash tbinom} applied on ${x}{x}$ is rendered as $\tbinom{x}{x}$

\texttt{\textbackslash tfrac} applied on ${x}{x}$ is rendered as $\tfrac{x}{x}$

\texttt{\textbackslash underset} applied on ${x}{x}$ is rendered as $\underset{x}{x}$

\section{ Group \texttt{fun\textunderscore ar2nb}}

\texttt{\textbackslash sideset} applied on ${_1^2}{_3^4}\sum$ is rendered as $\sideset{_1^2}{_3^4}\sum$

\section{ Group \texttt{fun\textunderscore infix}}

\texttt{\textbackslash atop} applied on $ x, y$ is rendered as $x\atop y$

\texttt{\textbackslash choose} applied on $ x, y$ is rendered as $x\choose y$

\texttt{\textbackslash over} applied on $ x, y$ is rendered as $x\over y$

\section{ Group \texttt{fun\textunderscore mhchem}}

\texttt{\textbackslash ce} is rendered as $\ce{x}$

\section{ Group \texttt{hline\textunderscore function}}

\texttt{\textbackslash hline} applied in a table is rendered as $\begin{matrix} x_{11} & x_{12} \\ \hline \end{matrix}$

\section{ Group \texttt{latex\textunderscore function\textunderscore names}}

\texttt{\textbackslash Pr} is rendered as $\Pr$

\texttt{\textbackslash arccos} is rendered as $\arccos$

\texttt{\textbackslash arcsin} is rendered as $\arcsin$

\texttt{\textbackslash arctan} is rendered as $\arctan$

\texttt{\textbackslash arg} is rendered as $\arg$

\texttt{\textbackslash cos} is rendered as $\cos$

\texttt{\textbackslash cosh} is rendered as $\cosh$

\texttt{\textbackslash cot} is rendered as $\cot$

\texttt{\textbackslash coth} is rendered as $\coth$

\texttt{\textbackslash csc} is rendered as $\csc$

\texttt{\textbackslash deg} is rendered as $\deg$

\texttt{\textbackslash det} is rendered as $\det$

\texttt{\textbackslash dim} is rendered as $\dim$

\texttt{\textbackslash exp} is rendered as $\exp$

\texttt{\textbackslash gcd} is rendered as $\gcd$

\texttt{\textbackslash hom} is rendered as $\hom$

\texttt{\textbackslash inf} is rendered as $\inf$

\texttt{\textbackslash ker} is rendered as $\ker$

\texttt{\textbackslash lg} is rendered as $\lg$

\texttt{\textbackslash lim} is rendered as $\lim$

\texttt{\textbackslash liminf} is rendered as $\liminf$

\texttt{\textbackslash limsup} is rendered as $\limsup$

\texttt{\textbackslash ln} is rendered as $\ln$

\texttt{\textbackslash log} is rendered as $\log$

\texttt{\textbackslash max} is rendered as $\max$

\texttt{\textbackslash min} is rendered as $\min$

\texttt{\textbackslash sec} is rendered as $\sec$

\texttt{\textbackslash sin} is rendered as $\sin$

\texttt{\textbackslash sinh} is rendered as $\sinh$

\texttt{\textbackslash sup} is rendered as $\sup$

\texttt{\textbackslash tan} is rendered as $\tan$

\texttt{\textbackslash tanh} is rendered as $\tanh$

\section{ Group \texttt{left\textunderscore function}}

\texttt{\textbackslash left} is rendered as $\left( \right.$

\section{ Group \texttt{mediawiki\textunderscore function\textunderscore names}}

\texttt{\textbackslash arccot} is rendered as $\operatorname{arccot} y$

\texttt{\textbackslash arccsc} is rendered as $\operatorname{arccsc} y$

\texttt{\textbackslash arcsec} is rendered as $\operatorname{arcsec} y$

\texttt{\textbackslash sen} is rendered as $\operatorname{sen} y$

\texttt{\textbackslash sgn} is rendered as $\operatorname{sgn} y$

\section{ Group \texttt{mhchem\textunderscore bond}}

\texttt{\textbackslash bond} is rendered as $\ce{\bond{-}}$

\section{ Group \texttt{mhchem\textunderscore macro\textunderscore 1p}}

\texttt{\textbackslash ce} is rendered as $\ce{\ce{x}}$

\texttt{\textbackslash mathbf} is rendered as $\ce{\mathbf{x}}$

\section{ Group \texttt{mhchem\textunderscore macro\textunderscore 2p}}

\texttt{\textbackslash frac} applied on ${x}{x}$ is rendered as $\ce{\frac{x}{x}}$

\texttt{\textbackslash overset} applied on ${x}{x}$ is rendered as $\ce{\overset{x}{x}}$

\texttt{\textbackslash underset} applied on ${x}{x}$ is rendered as $\ce{\underset{x}{x}}$

\section{ Group \texttt{mhchem\textunderscore macro\textunderscore 2pc}}

\texttt{\textbackslash color} is rendered as $\ce{\color{red}{red}}$

\section{ Group \texttt{mhchem\textunderscore macro\textunderscore 2pu}}

\texttt{\textbackslash underbrace} is rendered as $\ce{\underbrace{x}}$

\section{ Group \texttt{mhchem\textunderscore single\textunderscore macro}}

\texttt{\textbackslash Alpha} is rendered as $\ce{\Alpha}$

\texttt{\textbackslash Beta} is rendered as $\ce{\Beta}$

\texttt{\textbackslash Chi} is rendered as $\ce{\Chi}$

\texttt{\textbackslash Delta} is rendered as $\ce{\Delta}$

\texttt{\textbackslash Epsilon} is rendered as $\ce{\Epsilon}$

\texttt{\textbackslash Eta} is rendered as $\ce{\Eta}$

\texttt{\textbackslash Gamma} is rendered as $\ce{\Gamma}$

\texttt{\textbackslash Iota} is rendered as $\ce{\Iota}$

\texttt{\textbackslash Kappa} is rendered as $\ce{\Kappa}$

\texttt{\textbackslash Lambda} is rendered as $\ce{\Lambda}$

\texttt{\textbackslash Mu} is rendered as $\ce{\Mu}$

\texttt{\textbackslash Nu} is rendered as $\ce{\Nu}$

\texttt{\textbackslash Omega} is rendered as $\ce{\Omega}$

\texttt{\textbackslash Omicron} is rendered as $\ce{\Omicron}$

\texttt{\textbackslash Phi} is rendered as $\ce{\Phi}$

\texttt{\textbackslash Pi} is rendered as $\ce{\Pi}$

\texttt{\textbackslash Psi} is rendered as $\ce{\Psi}$

\texttt{\textbackslash Rho} is rendered as $\ce{\Rho}$

\texttt{\textbackslash Sigma} is rendered as $\ce{\Sigma}$

\texttt{\textbackslash Tau} is rendered as $\ce{\Tau}$

\texttt{\textbackslash Theta} is rendered as $\ce{\Theta}$

\texttt{\textbackslash Upsilon} is rendered as $\ce{\Upsilon}$

\texttt{\textbackslash Zeta} is rendered as $\ce{\Zeta}$

\texttt{\textbackslash alpha} is rendered as $\ce{\alpha}$

\texttt{\textbackslash approx} is rendered as $\ce{\approx}$

\texttt{\textbackslash beta} is rendered as $\ce{\beta}$

\texttt{\textbackslash ca} was never used. \newline  \url{https://phabricator.wikimedia.org/T323878}

\texttt{\textbackslash chi} is rendered as $\ce{\chi}$

\texttt{\textbackslash circ} is rendered as $\ce{\circ}$

\texttt{\textbackslash delta} is rendered as $\ce{\delta}$

\texttt{\textbackslash epsilon} is rendered as $\ce{\epsilon}$

\texttt{\textbackslash eta} is rendered as $\ce{\eta}$

\texttt{\textbackslash gamma} is rendered as $\ce{\gamma}$

\texttt{\textbackslash iota} is rendered as $\ce{\iota}$

\texttt{\textbackslash kappa} is rendered as $\ce{\kappa}$

\texttt{\textbackslash lambda} is rendered as $\ce{\lambda}$

\texttt{\textbackslash mu} is rendered as $\ce{\mu}$

\texttt{\textbackslash nu} is rendered as $\ce{\nu}$

\texttt{\textbackslash omega} is rendered as $\ce{\omega}$

\texttt{\textbackslash omicron} is rendered as $\ce{\omicron}$

\texttt{\textbackslash phi} is rendered as $\ce{\phi}$

\texttt{\textbackslash pi} is rendered as $\ce{\pi}$

\texttt{\textbackslash pm} is rendered as $\ce{\pm}$

\texttt{\textbackslash psi} is rendered as $\ce{\psi}$

\texttt{\textbackslash rho} is rendered as $\ce{\rho}$

\texttt{\textbackslash sigma} is rendered as $\ce{\sigma}$

\texttt{\textbackslash tau} is rendered as $\ce{\tau}$

\texttt{\textbackslash theta} is rendered as $\ce{\theta}$

\texttt{\textbackslash upsilon} is rendered as $\ce{\upsilon}$

\texttt{\textbackslash varepsilon} is rendered as $\ce{\varepsilon}$

\texttt{\textbackslash varkappa} is rendered as $\ce{\varkappa}$

\texttt{\textbackslash varphi} is rendered as $\ce{\varphi}$

\texttt{\textbackslash varpi} is rendered as $\ce{\varpi}$

\texttt{\textbackslash varrho} is rendered as $\ce{\varrho}$

\texttt{\textbackslash varsigma} is rendered as $\ce{\varsigma}$

\texttt{\textbackslash vartheta} is rendered as $\ce{\vartheta}$

\texttt{\textbackslash zeta} is rendered as $\ce{\zeta}$

\section{ Group \texttt{nullary\textunderscore macro}}

\texttt{\textbackslash And} is rendered as $\And$

\texttt{\textbackslash Bbbk} is rendered as $\Bbbk$

\texttt{\textbackslash Box} is rendered as $\Box$

\texttt{\textbackslash Bumpeq} is rendered as $\Bumpeq$

\texttt{\textbackslash Cap} is rendered as $\Cap$

\texttt{\textbackslash Cup} is rendered as $\Cup$

\texttt{\textbackslash Delta} is rendered as $\Delta$

\texttt{\textbackslash Diamond} is rendered as $\Diamond$

\texttt{\textbackslash Finv} is rendered as $\Finv$

\texttt{\textbackslash Game} is rendered as $\Game$

\texttt{\textbackslash Gamma} is rendered as $\Gamma$

\texttt{\textbackslash Im} is rendered as $\Im$

\texttt{\textbackslash Lambda} is rendered as $\Lambda$

\texttt{\textbackslash Leftarrow} is rendered as $\Leftarrow$

\texttt{\textbackslash Leftrightarrow} is rendered as $\Leftrightarrow$

\texttt{\textbackslash Lleftarrow} is rendered as $\Lleftarrow$

\texttt{\textbackslash Longleftarrow} is rendered as $\Longleftarrow$

\texttt{\textbackslash Longleftrightarrow} is rendered as $\Longleftrightarrow$

\texttt{\textbackslash Longrightarrow} is rendered as $\Longrightarrow$

\texttt{\textbackslash Lsh} is rendered as $\Lsh$

\texttt{\textbackslash Omega} is rendered as $\Omega$

\texttt{\textbackslash P} is rendered as $\P$

\texttt{\textbackslash Phi} is rendered as $\Phi$

\texttt{\textbackslash Pi} is rendered as $\Pi$

\texttt{\textbackslash Psi} is rendered as $\Psi$

\texttt{\textbackslash Re} is rendered as $\Re$

\texttt{\textbackslash Rightarrow} is rendered as $\Rightarrow$

\texttt{\textbackslash Rrightarrow} is rendered as $\Rrightarrow$

\texttt{\textbackslash Rsh} is rendered as $\Rsh$

\texttt{\textbackslash S} is rendered as $\S$

\texttt{\textbackslash Sigma} is rendered as $\Sigma$

\texttt{\textbackslash Subset} is rendered as $\Subset$

\texttt{\textbackslash Supset} is rendered as $\Supset$

\texttt{\textbackslash Theta} is rendered as $\Theta$

\texttt{\textbackslash Upsilon} is rendered as $\Upsilon$

\texttt{\textbackslash Vdash} is rendered as $\Vdash$

\texttt{\textbackslash Vvdash} is rendered as $\Vvdash$

\texttt{\textbackslash Xi} is rendered as $\Xi$

\texttt{\textbackslash aleph} is rendered as $\aleph$

\texttt{\textbackslash alpha} is rendered as $\alpha$

\texttt{\textbackslash amalg} is rendered as $\amalg$

\texttt{\textbackslash angle} is rendered as $\angle$

\texttt{\textbackslash approx} is rendered as $\approx$

\texttt{\textbackslash approxeq} is rendered as $\approxeq$

\texttt{\textbackslash ast} is rendered as $\ast$

\texttt{\textbackslash asymp} is rendered as $\asymp$

\texttt{\textbackslash backepsilon} is rendered as $\backepsilon$

\texttt{\textbackslash backprime} is rendered as $\backprime$

\texttt{\textbackslash backsim} is rendered as $\backsim$

\texttt{\textbackslash backsimeq} is rendered as $\backsimeq$

\texttt{\textbackslash barwedge} is rendered as $\barwedge$

\texttt{\textbackslash because} is rendered as $\because$

\texttt{\textbackslash beta} is rendered as $\beta$

\texttt{\textbackslash beth} is rendered as $\beth$

\texttt{\textbackslash between} is rendered as $\between$

\texttt{\textbackslash bigcap} is rendered as $\bigcap$

\texttt{\textbackslash bigcirc} is rendered as $\bigcirc$

\texttt{\textbackslash bigcup} is rendered as $\bigcup$

\texttt{\textbackslash bigodot} is rendered as $\bigodot$

\texttt{\textbackslash bigoplus} is rendered as $\bigoplus$

\texttt{\textbackslash bigotimes} is rendered as $\bigotimes$

\texttt{\textbackslash bigsqcup} is rendered as $\bigsqcup$

\texttt{\textbackslash bigstar} is rendered as $\bigstar$

\texttt{\textbackslash bigtriangledown} is rendered as $\bigtriangledown$

\texttt{\textbackslash bigtriangleup} is rendered as $\bigtriangleup$

\texttt{\textbackslash biguplus} is rendered as $\biguplus$

\texttt{\textbackslash bigvee} is rendered as $\bigvee$

\texttt{\textbackslash bigwedge} is rendered as $\bigwedge$

\texttt{\textbackslash blacklozenge} is rendered as $\blacklozenge$

\texttt{\textbackslash blacksquare} is rendered as $\blacksquare$

\texttt{\textbackslash blacktriangle} is rendered as $\blacktriangle$

\texttt{\textbackslash blacktriangledown} is rendered as $\blacktriangledown$

\texttt{\textbackslash blacktriangleleft} is rendered as $\blacktriangleleft$

\texttt{\textbackslash blacktriangleright} is rendered as $\blacktriangleright$

\texttt{\textbackslash bot} is rendered as $\bot$

\texttt{\textbackslash bowtie} is rendered as $\bowtie$

\texttt{\textbackslash boxdot} is rendered as $\boxdot$

\texttt{\textbackslash boxminus} is rendered as $\boxminus$

\texttt{\textbackslash boxplus} is rendered as $\boxplus$

\texttt{\textbackslash boxtimes} is rendered as $\boxtimes$

\texttt{\textbackslash bullet} is rendered as $\bullet$

\texttt{\textbackslash bumpeq} is rendered as $\bumpeq$

\texttt{\textbackslash cap} is rendered as $\cap$

\texttt{\textbackslash cdot} is rendered as $\cdot$

\texttt{\textbackslash cdots} is rendered as $\cdots$

\texttt{\textbackslash centerdot} is rendered as $\centerdot$

\texttt{\textbackslash checkmark} is rendered as $\checkmark$

\texttt{\textbackslash chi} is rendered as $\chi$

\texttt{\textbackslash circ} is rendered as $\circ$

\texttt{\textbackslash circeq} is rendered as $\circeq$

\texttt{\textbackslash circlearrowleft} is rendered as $\circlearrowleft$

\texttt{\textbackslash circlearrowright} is rendered as $\circlearrowright$

\texttt{\textbackslash circledS} is rendered as $\circledS$

\texttt{\textbackslash circledast} is rendered as $\circledast$

\texttt{\textbackslash circledcirc} is rendered as $\circledcirc$

\texttt{\textbackslash circleddash} is rendered as $\circleddash$

\texttt{\textbackslash clubsuit} is rendered as $\clubsuit$

\texttt{\textbackslash colon} is rendered as $\colon$

\texttt{\textbackslash complement} is rendered as $\complement$

\texttt{\textbackslash cong} is rendered as $\cong$

\texttt{\textbackslash coprod} is rendered as $\coprod$

\texttt{\textbackslash cup} is rendered as $\cup$

\texttt{\textbackslash curlyeqprec} is rendered as $\curlyeqprec$

\texttt{\textbackslash curlyeqsucc} is rendered as $\curlyeqsucc$

\texttt{\textbackslash curlyvee} is rendered as $\curlyvee$

\texttt{\textbackslash curlywedge} is rendered as $\curlywedge$

\texttt{\textbackslash curvearrowleft} is rendered as $\curvearrowleft$

\texttt{\textbackslash curvearrowright} is rendered as $\curvearrowright$

\texttt{\textbackslash dagger} is rendered as $\dagger$

\texttt{\textbackslash daleth} is rendered as $\daleth$

\texttt{\textbackslash dashv} is rendered as $\dashv$

\texttt{\textbackslash ddagger} is rendered as $\ddagger$

\texttt{\textbackslash ddots} is rendered as $\ddots$

\texttt{\textbackslash delta} is rendered as $\delta$

\texttt{\textbackslash diagdown} is rendered as $\diagdown$

\texttt{\textbackslash diagup} is rendered as $\diagup$

\texttt{\textbackslash diamond} is rendered as $\diamond$

\texttt{\textbackslash diamondsuit} is rendered as $\diamondsuit$

\texttt{\textbackslash digamma} is rendered as $\digamma$

\texttt{\textbackslash displaystyle} is rendered as $\displaystyle$

\texttt{\textbackslash div} is rendered as $\div$

\texttt{\textbackslash divideontimes} is rendered as $\divideontimes$

\texttt{\textbackslash doteq} is rendered as $\doteq$

\texttt{\textbackslash doteqdot} is rendered as $\doteqdot$

\texttt{\textbackslash dotplus} is rendered as $\dotplus$

\texttt{\textbackslash dots} is rendered as $\dots$

\texttt{\textbackslash dotsb} is rendered as $\dotsb$

\texttt{\textbackslash dotsc} is rendered as $\dotsc$

\texttt{\textbackslash dotsi} is rendered as $\dotsi$

\texttt{\textbackslash dotsm} is rendered as $\dotsm$

\texttt{\textbackslash dotso} is rendered as $\dotso$

\texttt{\textbackslash doublebarwedge} is rendered as $\doublebarwedge$

\texttt{\textbackslash downdownarrows} is rendered as $\downdownarrows$

\texttt{\textbackslash downharpoonleft} is rendered as $\downharpoonleft$

\texttt{\textbackslash downharpoonright} is rendered as $\downharpoonright$

\texttt{\textbackslash ell} is rendered as $\ell$

\texttt{\textbackslash emptyset} is rendered as $\emptyset$

\texttt{\textbackslash epsilon} is rendered as $\epsilon$

\texttt{\textbackslash eqcirc} is rendered as $\eqcirc$

\texttt{\textbackslash eqsim} is rendered as $\eqsim$

\texttt{\textbackslash eqslantgtr} is rendered as $\eqslantgtr$

\texttt{\textbackslash eqslantless} is rendered as $\eqslantless$

\texttt{\textbackslash equiv} is rendered as $\equiv$

\texttt{\textbackslash eta} is rendered as $\eta$

\texttt{\textbackslash eth} is rendered as $\eth$

\texttt{\textbackslash exists} is rendered as $\exists$

\texttt{\textbackslash fallingdotseq} is rendered as $\fallingdotseq$

\texttt{\textbackslash flat} is rendered as $\flat$

\texttt{\textbackslash forall} is rendered as $\forall$

\texttt{\textbackslash frown} is rendered as $\frown$

\texttt{\textbackslash gamma} is rendered as $\gamma$

\texttt{\textbackslash geq} is rendered as $\geq$

\texttt{\textbackslash geqq} is rendered as $\geqq$

\texttt{\textbackslash geqslant} is rendered as $\geqslant$

\texttt{\textbackslash gets} is rendered as $\gets$

\texttt{\textbackslash gg} is rendered as $\gg$

\texttt{\textbackslash ggg} is rendered as $\ggg$

\texttt{\textbackslash gimel} is rendered as $\gimel$

\texttt{\textbackslash gnapprox} is rendered as $\gnapprox$

\texttt{\textbackslash gneq} is rendered as $\gneq$

\texttt{\textbackslash gneqq} is rendered as $\gneqq$

\texttt{\textbackslash gnsim} is rendered as $\gnsim$

\texttt{\textbackslash gtrapprox} is rendered as $\gtrapprox$

\texttt{\textbackslash gtrdot} is rendered as $\gtrdot$

\texttt{\textbackslash gtreqless} is rendered as $\gtreqless$

\texttt{\textbackslash gtreqqless} is rendered as $\gtreqqless$

\texttt{\textbackslash gtrless} is rendered as $\gtrless$

\texttt{\textbackslash gtrsim} is rendered as $\gtrsim$

\texttt{\textbackslash gvertneqq} is rendered as $\gvertneqq$

\texttt{\textbackslash hbar} is rendered as $\hbar$

\texttt{\textbackslash heartsuit} is rendered as $\heartsuit$

\texttt{\textbackslash hookleftarrow} is rendered as $\hookleftarrow$

\texttt{\textbackslash hookrightarrow} is rendered as $\hookrightarrow$

\texttt{\textbackslash hslash} is rendered as $\hslash$

\texttt{\textbackslash iff} is rendered as $\iff$

\texttt{\textbackslash iiiint} is rendered as $\iiiint$

\texttt{\textbackslash iiint} is rendered as $\iiint$

\texttt{\textbackslash iint} is rendered as $\iint$

\texttt{\textbackslash imath} is rendered as $\imath$

\texttt{\textbackslash implies} is rendered as $\implies$

\texttt{\textbackslash in} is rendered as $\in$

\texttt{\textbackslash infty} is rendered as $\infty$

\texttt{\textbackslash injlim} is rendered as $\injlim$

\texttt{\textbackslash int} is rendered as $\int$

\texttt{\textbackslash intBar} is rendered as $\intBar$

\texttt{\textbackslash intbar} is rendered as $\intbar$

\texttt{\textbackslash intercal} is rendered as $\intercal$

\texttt{\textbackslash iota} is rendered as $\iota$

\texttt{\textbackslash jmath} is rendered as $\jmath$

\texttt{\textbackslash kappa} is rendered as $\kappa$

\texttt{\textbackslash lVert} is rendered as $\lVert$

\texttt{\textbackslash lambda} is rendered as $\lambda$

\texttt{\textbackslash land} is rendered as $\land$

\texttt{\textbackslash ldots} is rendered as $\ldots$

\texttt{\textbackslash leftarrow} is rendered as $\leftarrow$

\texttt{\textbackslash leftarrowtail} is rendered as $\leftarrowtail$

\texttt{\textbackslash leftharpoondown} is rendered as $\leftharpoondown$

\texttt{\textbackslash leftharpoonup} is rendered as $\leftharpoonup$

\texttt{\textbackslash leftleftarrows} is rendered as $\leftleftarrows$

\texttt{\textbackslash leftrightarrow} is rendered as $\leftrightarrow$

\texttt{\textbackslash leftrightarrows} is rendered as $\leftrightarrows$

\texttt{\textbackslash leftrightharpoons} is rendered as $\leftrightharpoons$

\texttt{\textbackslash leftrightsquigarrow} is rendered as $\leftrightsquigarrow$

\texttt{\textbackslash leftthreetimes} is rendered as $\leftthreetimes$

\texttt{\textbackslash leq} is rendered as $\leq$

\texttt{\textbackslash leqq} is rendered as $\leqq$

\texttt{\textbackslash leqslant} is rendered as $\leqslant$

\texttt{\textbackslash lessapprox} is rendered as $\lessapprox$

\texttt{\textbackslash lessdot} is rendered as $\lessdot$

\texttt{\textbackslash lesseqgtr} is rendered as $\lesseqgtr$

\texttt{\textbackslash lesseqqgtr} is rendered as $\lesseqqgtr$

\texttt{\textbackslash lessgtr} is rendered as $\lessgtr$

\texttt{\textbackslash lesssim} is rendered as $\lesssim$

\texttt{\textbackslash limits} is rendered for example as $\mathop\cap\limits_a^b$

\texttt{\textbackslash ll} is rendered as $\ll$

\texttt{\textbackslash lll} is rendered as $\lll$

\texttt{\textbackslash lnapprox} is rendered as $\lnapprox$

\texttt{\textbackslash lneq} is rendered as $\lneq$

\texttt{\textbackslash lneqq} is rendered as $\lneqq$

\texttt{\textbackslash lnot} is rendered as $\lnot$

\texttt{\textbackslash lnsim} is rendered as $\lnsim$





\texttt{\textbackslash longleftarrow} is rendered as $\longleftarrow$

\texttt{\textbackslash longleftrightarrow} is rendered as $\longleftrightarrow$



\texttt{\textbackslash longmapsto} is rendered as $\longmapsto$

\texttt{\textbackslash longrightarrow} is rendered as $\longrightarrow$



\texttt{\textbackslash looparrowleft} is rendered as $\looparrowleft$

\texttt{\textbackslash looparrowright} is rendered as $\looparrowright$

\texttt{\textbackslash lor} is rendered as $\lor$

\texttt{\textbackslash lozenge} is rendered as $\lozenge$

\texttt{\textbackslash ltimes} is rendered as $\ltimes$

\texttt{\textbackslash lvertneqq} is rendered as $\lvertneqq$

\texttt{\textbackslash mapsto} is rendered as $\mapsto$

\texttt{\textbackslash measuredangle} is rendered as $\measuredangle$

\texttt{\textbackslash mho} is rendered as $\mho$

\texttt{\textbackslash mid} is rendered as $\mid$

\texttt{\textbackslash mod} is rendered as $\mod$

\texttt{\textbackslash models} is rendered as $\models$

\texttt{\textbackslash mp} is rendered as $\mp$

\texttt{\textbackslash mu} is rendered as $\mu$

\texttt{\textbackslash multimap} is rendered as $\multimap$

\texttt{\textbackslash nLeftarrow} is rendered as $\nLeftarrow$

\texttt{\textbackslash nLeftrightarrow} is rendered as $\nLeftrightarrow$

\texttt{\textbackslash nRightarrow} is rendered as $\nRightarrow$

\texttt{\textbackslash nVDash} is rendered as $\nVDash$

\texttt{\textbackslash nVdash} is rendered as $\nVdash$

\texttt{\textbackslash nabla} is rendered as $\nabla$

\texttt{\textbackslash natural} is rendered as $\natural$

\texttt{\textbackslash ncong} is rendered as $\ncong$

\texttt{\textbackslash nearrow} is rendered as $\nearrow$

\texttt{\textbackslash neg} is rendered as $\neg$

\texttt{\textbackslash neq} is rendered as $\neq$

\texttt{\textbackslash nexists} is rendered as $\nexists$

\texttt{\textbackslash ngeq} is rendered as $\ngeq$

\texttt{\textbackslash ngeqq} is rendered as $\ngeqq$

\texttt{\textbackslash ngeqslant} is rendered as $\ngeqslant$

\texttt{\textbackslash ngtr} is rendered as $\ngtr$

\texttt{\textbackslash ni} is rendered as $\ni$

\texttt{\textbackslash nleftarrow} is rendered as $\nleftarrow$

\texttt{\textbackslash nleftrightarrow} is rendered as $\nleftrightarrow$

\texttt{\textbackslash nleq} is rendered as $\nleq$

\texttt{\textbackslash nleqq} is rendered as $\nleqq$

\texttt{\textbackslash nleqslant} is rendered as $\nleqslant$

\texttt{\textbackslash nless} is rendered as $\nless$

\texttt{\textbackslash nmid} is rendered as $\nmid$

\texttt{\textbackslash nolimits} is rendered for example as $\mathop\cap\nolimits_a^b$

\texttt{\textbackslash not} is rendered as $\not$

\texttt{\textbackslash notin} is rendered as $\notin$

\texttt{\textbackslash nparallel} is rendered as $\nparallel$

\texttt{\textbackslash nprec} is rendered as $\nprec$

\texttt{\textbackslash npreceq} is rendered as $\npreceq$

\texttt{\textbackslash nrightarrow} is rendered as $\nrightarrow$

\texttt{\textbackslash nshortmid} is rendered as $\nshortmid$

\texttt{\textbackslash nshortparallel} is rendered as $\nshortparallel$

\texttt{\textbackslash nsim} is rendered as $\nsim$

\texttt{\textbackslash nsubseteq} is rendered as $\nsubseteq$

\texttt{\textbackslash nsubseteqq} is rendered as $\nsubseteqq$

\texttt{\textbackslash nsucc} is rendered as $\nsucc$

\texttt{\textbackslash nsucceq} is rendered as $\nsucceq$

\texttt{\textbackslash nsupseteq} is rendered as $\nsupseteq$

\texttt{\textbackslash nsupseteqq} is rendered as $\nsupseteqq$

\texttt{\textbackslash ntriangleleft} is rendered as $\ntriangleleft$

\texttt{\textbackslash ntrianglelefteq} is rendered as $\ntrianglelefteq$

\texttt{\textbackslash ntriangleright} is rendered as $\ntriangleright$

\texttt{\textbackslash ntrianglerighteq} is rendered as $\ntrianglerighteq$

\texttt{\textbackslash nu} is rendered as $\nu$

\texttt{\textbackslash nvDash} is rendered as $\nvDash$

\texttt{\textbackslash nvdash} is rendered as $\nvdash$

\texttt{\textbackslash nwarrow} is rendered as $\nwarrow$

\texttt{\textbackslash odot} is rendered as $\odot$

\texttt{\textbackslash oiiint} is rendered as $\oiiint$

\texttt{\textbackslash oiint} is rendered as $\oiint$

\texttt{\textbackslash oint} is rendered as $\oint$

\texttt{\textbackslash ointctrclockwise} is rendered as $\ointctrclockwise$

\texttt{\textbackslash omega} is rendered as $\omega$

\texttt{\textbackslash ominus} is rendered as $\ominus$

\texttt{\textbackslash oplus} is rendered as $\oplus$

\texttt{\textbackslash oslash} is rendered as $\oslash$

\texttt{\textbackslash otimes} is rendered as $\otimes$

\texttt{\textbackslash parallel} is rendered as $\parallel$

\texttt{\textbackslash partial} is rendered as $\partial$

\texttt{\textbackslash perp} is rendered as $\perp$

\texttt{\textbackslash phi} is rendered as $\phi$

\texttt{\textbackslash pi} is rendered as $\pi$

\texttt{\textbackslash pitchfork} is rendered as $\pitchfork$

\texttt{\textbackslash pm} is rendered as $\pm$

\texttt{\textbackslash prec} is rendered as $\prec$

\texttt{\textbackslash precapprox} is rendered as $\precapprox$

\texttt{\textbackslash preccurlyeq} is rendered as $\preccurlyeq$

\texttt{\textbackslash preceq} is rendered as $\preceq$

\texttt{\textbackslash precnapprox} is rendered as $\precnapprox$

\texttt{\textbackslash precneqq} is rendered as $\precneqq$

\texttt{\textbackslash precnsim} is rendered as $\precnsim$

\texttt{\textbackslash precsim} is rendered as $\precsim$

\texttt{\textbackslash prime} is rendered as $\prime$

\texttt{\textbackslash prod} is rendered as $\prod$

\texttt{\textbackslash projlim} is rendered as $\projlim$

\texttt{\textbackslash propto} is rendered as $\propto$

\texttt{\textbackslash psi} is rendered as $\psi$

\texttt{\textbackslash qquad} is rendered as $\qquad$

\texttt{\textbackslash quad} is rendered as $\quad$

\texttt{\textbackslash rVert} is rendered as $\rVert$

\texttt{\textbackslash rho} is rendered as $\rho$

\texttt{\textbackslash rightarrow} is rendered as $\rightarrow$

\texttt{\textbackslash rightarrowtail} is rendered as $\rightarrowtail$

\texttt{\textbackslash rightharpoondown} is rendered as $\rightharpoondown$

\texttt{\textbackslash rightharpoonup} is rendered as $\rightharpoonup$

\texttt{\textbackslash rightleftarrows} is rendered as $\rightleftarrows$

\texttt{\textbackslash rightrightarrows} is rendered as $\rightrightarrows$

\texttt{\textbackslash rightsquigarrow} is rendered as $\rightsquigarrow$

\texttt{\textbackslash rightthreetimes} is rendered as $\rightthreetimes$

\texttt{\textbackslash risingdotseq} is rendered as $\risingdotseq$

\texttt{\textbackslash rtimes} is rendered as $\rtimes$

\texttt{\textbackslash scriptscriptstyle} is rendered as $\scriptscriptstyle$

\texttt{\textbackslash scriptstyle} is rendered as $\scriptstyle$

\texttt{\textbackslash searrow} is rendered as $\searrow$

\texttt{\textbackslash setminus} is rendered as $\setminus$

\texttt{\textbackslash sharp} is rendered as $\sharp$

\texttt{\textbackslash shortmid} is rendered as $\shortmid$

\texttt{\textbackslash shortparallel} is rendered as $\shortparallel$

\texttt{\textbackslash sigma} is rendered as $\sigma$

\texttt{\textbackslash sim} is rendered as $\sim$

\texttt{\textbackslash simeq} is rendered as $\simeq$

\texttt{\textbackslash smallfrown} is rendered as $\smallfrown$

\texttt{\textbackslash smallsetminus} is rendered as $\smallsetminus$

\texttt{\textbackslash smallsmile} is rendered as $\smallsmile$

\texttt{\textbackslash smile} is rendered as $\smile$

\texttt{\textbackslash spadesuit} is rendered as $\spadesuit$

\texttt{\textbackslash sphericalangle} is rendered as $\sphericalangle$

\texttt{\textbackslash sqcap} is rendered as $\sqcap$

\texttt{\textbackslash sqcup} is rendered as $\sqcup$

\texttt{\textbackslash sqsubset} is rendered as $\sqsubset$

\texttt{\textbackslash sqsubseteq} is rendered as $\sqsubseteq$

\texttt{\textbackslash sqsupset} is rendered as $\sqsupset$

\texttt{\textbackslash sqsupseteq} is rendered as $\sqsupseteq$

\texttt{\textbackslash square} is rendered as $\square$

\texttt{\textbackslash star} is rendered as $\star$

\texttt{\textbackslash subset} is rendered as $\subset$

\texttt{\textbackslash subseteq} is rendered as $\subseteq$

\texttt{\textbackslash subseteqq} is rendered as $\subseteqq$

\texttt{\textbackslash subsetneq} is rendered as $\subsetneq$

\texttt{\textbackslash subsetneqq} is rendered as $\subsetneqq$

\texttt{\textbackslash succ} is rendered as $\succ$

\texttt{\textbackslash succapprox} is rendered as $\succapprox$

\texttt{\textbackslash succcurlyeq} is rendered as $\succcurlyeq$

\texttt{\textbackslash succeq} is rendered as $\succeq$

\texttt{\textbackslash succnapprox} is rendered as $\succnapprox$

\texttt{\textbackslash succneqq} is rendered as $\succneqq$

\texttt{\textbackslash succnsim} is rendered as $\succnsim$

\texttt{\textbackslash succsim} is rendered as $\succsim$

\texttt{\textbackslash sum} is rendered as $\sum$

\texttt{\textbackslash supset} is rendered as $\supset$

\texttt{\textbackslash supseteq} is rendered as $\supseteq$

\texttt{\textbackslash supseteqq} is rendered as $\supseteqq$

\texttt{\textbackslash supsetneq} is rendered as $\supsetneq$

\texttt{\textbackslash supsetneqq} is rendered as $\supsetneqq$

\texttt{\textbackslash surd} is rendered as $\surd$

\texttt{\textbackslash swarrow} is rendered as $\swarrow$

\texttt{\textbackslash tau} is rendered as $\tau$

\texttt{\textbackslash textstyle} is rendered as $\textstyle$

\texttt{\textbackslash therefore} is rendered as $\therefore$

\texttt{\textbackslash theta} is rendered as $\theta$

\texttt{\textbackslash thickapprox} is rendered as $\thickapprox$

\texttt{\textbackslash thicksim} is rendered as $\thicksim$

\texttt{\textbackslash times} is rendered as $\times$

\texttt{\textbackslash to} is rendered as $\to$

\texttt{\textbackslash top} is rendered as $\top$

\texttt{\textbackslash triangle} is rendered as $\triangle$

\texttt{\textbackslash triangledown} is rendered as $\triangledown$

\texttt{\textbackslash triangleleft} is rendered as $\triangleleft$

\texttt{\textbackslash trianglelefteq} is rendered as $\trianglelefteq$

\texttt{\textbackslash triangleq} is rendered as $\triangleq$

\texttt{\textbackslash triangleright} is rendered as $\triangleright$

\texttt{\textbackslash trianglerighteq} is rendered as $\trianglerighteq$



\texttt{\textbackslash upharpoonleft} is rendered as $\upharpoonleft$

\texttt{\textbackslash upharpoonright} is rendered as $\upharpoonright$

\texttt{\textbackslash uplus} is rendered as $\uplus$

\texttt{\textbackslash upsilon} is rendered as $\upsilon$

\texttt{\textbackslash upuparrows} is rendered as $\upuparrows$

\texttt{\textbackslash vDash} is rendered as $\vDash$

\texttt{\textbackslash varDelta} is rendered as $\varDelta$

\texttt{\textbackslash varGamma} is rendered as $\varGamma$

\texttt{\textbackslash varLambda} is rendered as $\varLambda$

\texttt{\textbackslash varOmega} is rendered as $\varOmega$

\texttt{\textbackslash varPhi} is rendered as $\varPhi$

\texttt{\textbackslash varPi} is rendered as $\varPi$

\texttt{\textbackslash varSigma} is rendered as $\varSigma$

\texttt{\textbackslash varTheta} is rendered as $\varTheta$

\texttt{\textbackslash varUpsilon} is rendered as $\varUpsilon$

\texttt{\textbackslash varXi} is rendered as $\varXi$

\texttt{\textbackslash varepsilon} is rendered as $\varepsilon$

\texttt{\textbackslash varinjlim} is rendered as $\varinjlim$

\texttt{\textbackslash varkappa} is rendered as $\varkappa$

\texttt{\textbackslash varliminf} is rendered as $\varliminf$

\texttt{\textbackslash varlimsup} is rendered as $\varlimsup$

\texttt{\textbackslash varnothing} is rendered as $\varnothing$

\texttt{\textbackslash varointclockwise} is rendered as $\varointclockwise$

\texttt{\textbackslash varphi} is rendered as $\varphi$

\texttt{\textbackslash varpi} is rendered as $\varpi$

\texttt{\textbackslash varprojlim} is rendered as $\varprojlim$

\texttt{\textbackslash varpropto} is rendered as $\varpropto$

\texttt{\textbackslash varrho} is rendered as $\varrho$

\texttt{\textbackslash varsigma} is rendered as $\varsigma$

\texttt{\textbackslash varsubsetneq} is rendered as $\varsubsetneq$

\texttt{\textbackslash varsubsetneqq} is rendered as $\varsubsetneqq$

\texttt{\textbackslash varsupsetneq} is rendered as $\varsupsetneq$

\texttt{\textbackslash varsupsetneqq} is rendered as $\varsupsetneqq$

\texttt{\textbackslash vartheta} is rendered as $\vartheta$

\texttt{\textbackslash vartriangle} is rendered as $\vartriangle$

\texttt{\textbackslash vartriangleleft} is rendered as $\vartriangleleft$

\texttt{\textbackslash vartriangleright} is rendered as $\vartriangleright$

\texttt{\textbackslash vdash} is rendered as $\vdash$

\texttt{\textbackslash vdots} is rendered as $\vdots$

\texttt{\textbackslash vee} is rendered as $\vee$

\texttt{\textbackslash veebar} is rendered as $\veebar$

\texttt{\textbackslash vline} is rendered as $\vline$

\texttt{\textbackslash wedge} is rendered as $\wedge$

\texttt{\textbackslash wp} is rendered as $\wp$

\texttt{\textbackslash wr} is rendered as $\wr$

\texttt{\textbackslash xi} is rendered as $\xi$

\texttt{\textbackslash zeta} is rendered as $\zeta$

\section{ Group \texttt{nullary\textunderscore macro\textunderscore in\textunderscore mbox}}

\texttt{\textbackslash AA} is rendered as $\AA$

\texttt{\textbackslash Coppa} is rendered as $\Coppa$

\texttt{\textbackslash Digamma} is rendered as $\Digamma$

\texttt{\textbackslash Koppa} is rendered as $\Koppa$

\texttt{\textbackslash Sampi} is rendered as $\Sampi$

\texttt{\textbackslash Stigma} is rendered as $\Stigma$

\texttt{\textbackslash coppa} is rendered as $\coppa$

\texttt{\textbackslash euro} is rendered as $\euro$

\texttt{\textbackslash geneuro} is rendered as $\geneuro$

\texttt{\textbackslash geneuronarrow} is rendered as $\geneuronarrow$

\texttt{\textbackslash geneurowide} is rendered as $\geneurowide$

\texttt{\textbackslash koppa} is rendered as $\koppa$

\texttt{\textbackslash officialeuro} is rendered as $\officialeuro$

\texttt{\textbackslash sampi} is rendered as $\sampi$

\texttt{\textbackslash stigma} is rendered as $\stigma$

\texttt{\textbackslash textvisiblespace} is rendered as $\textvisiblespace$

\texttt{\textbackslash varstigma} is rendered as $\varstigma$

\section{ Group \texttt{other\textunderscore delimiters1}}

\texttt{\textbackslash Downarrow} is rendered as $\Downarrow$

\texttt{\textbackslash Uparrow} is rendered as $\Uparrow$

\texttt{\textbackslash Updownarrow} is rendered as $\Updownarrow$

\texttt{\textbackslash Vert} is rendered as $\Vert$

\texttt{\textbackslash backslash} is rendered as $\backslash$

\texttt{\textbackslash downarrow} is rendered as $\downarrow$

\texttt{\textbackslash langle} is rendered as $\langle$

\texttt{\textbackslash lbrace} is rendered as $\lbrace$

\texttt{\textbackslash lbrack} is rendered as $\lbrack$

\texttt{\textbackslash lceil} is rendered as $\lceil$

\texttt{\textbackslash lfloor} is rendered as $\lfloor$

\texttt{\textbackslash llcorner} is rendered as $\llcorner$

\texttt{\textbackslash lrcorner} is rendered as $\lrcorner$

\texttt{\textbackslash rangle} is rendered as $\rangle$

\texttt{\textbackslash rbrace} is rendered as $\rbrace$

\texttt{\textbackslash rbrack} is rendered as $\rbrack$

\texttt{\textbackslash rceil} is rendered as $\rceil$

\texttt{\textbackslash rfloor} is rendered as $\rfloor$

\texttt{\textbackslash rightleftharpoons} is rendered as $\rightleftharpoons$

\texttt{\textbackslash twoheadleftarrow} is rendered as $\twoheadleftarrow$

\texttt{\textbackslash twoheadrightarrow} is rendered as $\twoheadrightarrow$

\texttt{\textbackslash ulcorner} is rendered as $\ulcorner$

\texttt{\textbackslash uparrow} is rendered as $\uparrow$

\texttt{\textbackslash updownarrow} is rendered as $\updownarrow$

\texttt{\textbackslash urcorner} is rendered as $\urcorner$

\texttt{\textbackslash vert} is rendered as $\vert$

\section{ Group \texttt{other\textunderscore delimiters2}}

\texttt{\textbackslash Darr} is rendered as $\Darr$

\texttt{\textbackslash Uarr} is rendered as $\Uarr$

\texttt{\textbackslash dArr} is rendered as $\dArr$

\texttt{\textbackslash darr} is rendered as $\darr$

\texttt{\textbackslash lang} is rendered as $\lang$

\texttt{\textbackslash rang} is rendered as $\rang$

\texttt{\textbackslash uArr} is rendered as $\uArr$

\texttt{\textbackslash uarr} is rendered as $\uarr$

\section{ Group \texttt{right\textunderscore function}}

\texttt{\textbackslash right} is rendered as $\left. \right)$

\end{document}
